\documentclass[12pt,a4paper]{article}

% Preamble - Package imports
\usepackage[utf8]{inputenc}
\usepackage[T1]{fontenc}
\usepackage{amsmath,amssymb,amsthm}
\usepackage{graphicx}
\usepackage{xcolor}
\usepackage{hyperref}
\usepackage{listings}
\usepackage{tikz}
\usepackage{algorithm}
\usepackage{algorithmic}
\usepackage{natbib}
\usepackage{geometry}

% Document settings
\geometry{margin=1in}
\hypersetup{
    colorlinks=true,
    linkcolor=blue,
    filecolor=magenta,
    urlcolor=cyan,
}

% Custom commands
\newcommand{\comphy}{\textsc{CoMPhy Gruvbox}}
\newcommand{\R}{\mathbb{R}}
\newcommand{\N}{\mathbb{N}}
\newcommand{\Z}{\mathbb{Z}}
\newcommand{\norm}[1]{\left\lVert#1\right\rVert}
\renewcommand{\vec}[1]{\mathbf{#1}}

% Theorem environments
\newtheorem{theorem}{Theorem}[section]
\newtheorem{lemma}[theorem]{Lemma}
\newtheorem{proposition}[theorem]{Proposition}
\newtheorem{corollary}[theorem]{Corollary}
\theoremstyle{definition}
\newtheorem{definition}[theorem]{Definition}
\newtheorem{example}[theorem]{Example}
\theoremstyle{remark}
\newtheorem{remark}[theorem]{Remark}

% Code listing style
\lstdefinestyle{comphy}{
    basicstyle=\ttfamily\small,
    keywordstyle=\color{red}\bfseries,
    stringstyle=\color{green},
    commentstyle=\color{gray}\itshape,
    numbers=left,
    numberstyle=\tiny\color{gray},
    breaklines=true,
    frame=single,
    backgroundcolor=\color{gray!10}
}

\title{{\Huge\comphy{} Theme}\\[0.5em]
       \Large A LaTeX Demonstration Document}
\author{Vatsal Sanjay\\
        \texttt{vatsal@example.com}\\[0.5em]
        Department of Computer Science\\
        University of Examples}
\date{\today}

\begin{document}

\maketitle

\begin{abstract}
This document demonstrates the \LaTeX{} syntax highlighting capabilities of the \comphy{} theme. We explore various \LaTeX{} constructs including mathematical formulas, environments, custom commands, and advanced typesetting features. The theme provides excellent contrast and readability for all \LaTeX{} elements, making document preparation more enjoyable and efficient.
\end{abstract}

\tableofcontents
\newpage

\section{Introduction}
\label{sec:intro}

Welcome to this comprehensive \LaTeX{} demonstration showcasing the \comphy{} theme's syntax highlighting capabilities.

\subsection{Document Structure}

A typical \LaTeX{} document consists of:
\begin{itemize}
    \item A \textbf{preamble} containing package imports and settings
    \item The \texttt{document} environment containing the actual content
    \item Various \emph{sections} and \emph{subsections} for organization
\end{itemize}

\section{Mathematics}
\label{sec:math}

\subsection{Inline Mathematics}

Inline math mode is entered using \verb|$...$|. The quadratic formula is $x = \frac{-b \pm \sqrt{b^2 - 4ac}}{2a}$, and Euler's identity is $e^{i\pi} + 1 = 0$.

\subsection{Display Mathematics}

\begin{equation}
    \label{eq:gaussian}
    f(x) = \frac{1}{\sigma\sqrt{2\pi}} \exp\left(-\frac{(x-\mu)^2}{2\sigma^2}\right)
\end{equation}

The Gaussian distribution in Equation~\eqref{eq:gaussian} is fundamental in probability theory.

\begin{theorem}[Fundamental Theorem of Calculus]
\label{thm:ftc}
If $f$ is continuous on $[a,b]$ and $F$ is an antiderivative of $f$ on $[a,b]$, then:
\[
    \int_a^b f(x)\,dx = F(b) - F(a)
\]
\end{theorem}

\begin{proof}
Let $F'(x) = f(x)$ for all $x \in [a,b]$. By the definition of the definite integral and the mean value theorem, we can show that the integral equals $F(b) - F(a)$.
\end{proof}

\subsection{Matrix Operations}

\begin{equation}
    \vec{A} = \begin{pmatrix}
        a_{11} & a_{12} & \cdots & a_{1n} \\
        a_{21} & a_{22} & \cdots & a_{2n} \\
        \vdots & \vdots & \ddots & \vdots \\
        a_{m1} & a_{m2} & \cdots & a_{mn}
    \end{pmatrix}
\end{equation}

\section{Conclusion}

This document has demonstrated the comprehensive \LaTeX{} syntax highlighting capabilities of the \comphy{} theme.

\end{document}
